
\subsection{Aufgabe 1}

F"ur die Nutzung des Port4 als Ausgang werden folgende Register ben\"otigt:
\begin{itemize}
    \item P4SEL = 0x00;
    \item P4DIR
    \item P4OUT
    \item P4IN
\end{itemize}

Die obere Abbildung zeigt die innere Struktur einer Portleitung.

Alle Leitungen des Port4 werden als Ausgang genutzt. Die LED leuchten, wenn die zugeh\"orige Portleitung den Wert ``0'' hat. 
	

Erl\"autern Sie die folgenden Befehle und deren Auswirkung in der Reihenfolge ihrer Abarbeitung auf die LED, den Beeper und den Wert der Variablen a (unsigned char).
\begin{lstlisting}
     #define LEDRT (0x01)   // Bitmuster 0000 0001 fuer die Steuerung der Roten LED (also auf Portleitung 0 eines Ports) 
     P4DIR = 0x00;          // Der gesamte Port wird auf Eingang umgestellt, Konsequenz P4IN wird durch LEDs 0000 0111
     a = 10;                // lokale Variable bekommt den wert 10 = 0000 1010
     P4OUT = a;             // Ausgabewert der Portleitung 1 und 3 (Gelb und Beeper) werden auf 1 gesetzt ausgestellt, allerdings ist die Richtung noch auf Eingang gestellt
     P4OUT = 0x01;          // Register wird wieder \"uberschrieben, nur Rot wird ausgestellt 
     P4DIR = 0x07;          // Alle Portleitungen f\"ur die LEDs werden auf Ausgang umgestellt
     a = P4IN & 0x07;       // 
     P4OUT &= 0x00;
     P4OUT |= 0x01;
     P4OUT |= LEDRT;
     P4OUT &= ~LEDRT;
     P3OUT ^= LEDRT;
\end{lstlisting}

Folgende Belegungen der Portleitungen auf Basis der Schaltung werden vorausgesetzt:
\begin{itemize}
    \item P4.0 LED rot
    \item P4.1 LED gelb
    \item P4.2 LED gr�n
    \item P4.3 Beeper
\end{itemize}



Schreiben Sie ein kleines Programm, welches den Durchlauf einer Ampelsignalsequenz mit den Phasen (rt,rt-gb,gr,gb,rt) simuliert. Nutzen Sie dazu die bereitgestellte Funktion f�r eine Zeitschleife.  



% \subsubsection{Aufgabenstellung}


% \subsubsection{L\"osung}

